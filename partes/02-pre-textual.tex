
%----------------------------------------------------------------------------------------%
% I N Í C I O   D O   D O C U M E N T O - P R É   T E X T U A L

%----------------------------------------------------------------------------------------%






\begin{center}
	\setstretch{1.75}
	\textbf{\LARGE Cartografia dos Sotaques Brasileiros:}\\[0.5em]
	\textbf{\LARGE Uma proposta de identificação e organização de falantes por código DDD}\\[0.5em]
\end{center}

\vspace{2em}

\begin{center}
	\setstretch{1}
	\textbf{Jonas Ernesto Poli}\\
	Mestrando em Ciência, Tecnologia e Sociedade pela Universidade\\
	Federal de São Carlos, Brasil.\\
	E-mail: \href{mailto:jonaspoli@ufscar.br}{jonaspoli@ufscar.br}
\end{center}

\vspace{2em}

\begin{center}
	\setstretch{1}
	\textbf{Luís Fernando Soares Zuin}\\
    Doutor em Engenharia de Produção\\
	Federal de São Carlos, Brasil. Professora da Universidade Federal de\\
	São Carlos, Brasil.\\
	E-mail: \href{mailto:zuin@ufscar.br}{zuin@ufscar.br}
\end{center}
	
	
	%------------------------------------------------------------------------------------%
	% R E S U M O   N O   I D I O M A   D O   T E X T O
	%
	% OBRIGATÓRIO. Elaborado conforme a NBR 6028. 
	% Deve ser redigido em um só parágrafo contendo de 150 a 500 palavras e ressaltar: objetivo, método, resultados e as principais conclusões.
	% Após o resumo, são listadas palavras-chave relacionadas à temática do trabalho, separadas entre si por ponto e virgula (;) e finalizadas por ponto final.  (NBR 6028)
	%------------------------------------------------------------------------------------%
	
	\begin{resumo}
		\setstretch{1}

Este artigo propõe a utilização do mapa de Discagem Direta à Distância (DDD) como referência para o mapeamento dos dialetos brasileiros, visando o desenvolvimento de tecnologias assistivas que se adaptem às variações linguísticas regionais. Partindo da divisão dialetal histórica de Antenor Nascentes (1922) e de estudos contemporâneos do Atlas Linguístico do Brasil (ALiB), estabelecemos correlações entre os códigos de área telefônicos e os dialetos identificados no território nacional. A fundamentação teórica baseia-se no dialogismo de Bakhtin e na pedagogia libertadora de Paulo Freire, compreendendo a linguagem como fenômeno social e a comunicação como prática dialógica. A metodologia envolveu o mapeamento sistemático entre DDDs e dialetos, identificando sobreposições e lacunas. Os resultados revelam que, apesar das limitações, o sistema de DDD oferece uma infraestrutura prática para o desenvolvimento de tecnologias que respeitem a diversidade linguística brasileira, especialmente no contexto da comunicação rural e da ATER digital participativa. Concluímos que esta abordagem representa um passo significativo para a criação de interfaces tecnológicas que reconheçam e valorizem os modos de falar locais, promovendo inclusão digital e respeito à identidade cultural.
\\[0.5em]

		
		\vspace*{0.5cm}
		
		\noindent\textbf{{Palavras-Chave: }}  Dialetos brasileiros; DDD; Dialogismo; Tecnologias assistivas; Comunicação rural.

		
	\end{resumo}
	
	%------------------------------------------------------------------------------------%
	% A B S T R A C T :   R E S U M O   N O   I D I O M A   E S T R A N G E I R O
	%
	% OBRIGATÓRIO. Elaborado com as mesmas características do resumo em língua portuguesa.
	% Se redigido em inglês-ABSTRACT, em castelhano-RESUMEN, em francês-RÉSUMÉ.
	% Após o resumo, são listadas palavras-chave relacionadas à temática do trabalho no 
	% idioma escolhido. Se redigido em inglês - KEYWORDS, em espanhol - PALABRAS CLAVES,
	% em francês - MOTS-CLÉS.
	% TEXTO ATUAL 
	%ZUSAMMENFASSUNG (alemão)
	%RIASSUNTO (italiano)
	
	%Keywords (inglês)
	%Palabras clave (espanhol)
	%Mot-clé (francês)
	%Stichwörter (alemão)
	%Parole chiave (italiano)
	%------------------------------------------------------------------------------------%
	\begin{center}
		\setstretch{1}
		\textbf{\large Cartography of Brazilian Accents: a proposal for identifying and organising speakers by DDD area code}
	\end{center}	
	\begin{resumo}[Abstract] % Substitua 'Abstract' pela palavra no idioma desejado, caso precise.
		\setstretch{1}
This article proposes the use of the Direct Distance Dialing (DDD) map as a reference for mapping Brazilian dialects, aiming at the development of assistive technologies that adapt to regional linguistic variations. Starting from Antenor Nascentes' historical dialectal division (1922) and contemporary studies from the Linguistic Atlas of Brazil (ALiB), we establish correlations between telephone area codes and dialects identified in the national territory. The theoretical foundation is based on Bakhtin's dialogism and
Paulo Freire's liberating pedagogy, understanding language as a social phenomenon and communication as a dialogical practice. The methodology involved systematic mapping between DDDs and dialects, identifying overlaps and gaps. The results reveal that, despite limitations, the DDD system offers a practical infrastructure for developing technologies that respect Brazilian linguistic diversity, especially in the context of rural communication and participatory digital ATER. We conclude that this approach represents a significant step towards creating technological interfaces that recognize and value local ways of speaking, promoting digital inclusion and respect for cultural identity.
\\[0.5em]

		
		\vspace*{0.5cm}
		
		%Subistitua 'Keywords' pela palavra no idioma desejado, caso precise.
		\noindent\textbf{{Keywords: }} Brazilian dialects; DDD; Dialogism; Assistive technologies; Rural communication.
		
		
	\end{resumo}
	

	
	

	
