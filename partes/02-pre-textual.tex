
%----------------------------------------------------------------------------------------%
% I N Í C I O   D O   D O C U M E N T O - P R É   T E X T U A L

%----------------------------------------------------------------------------------------%






\begin{center}
	\setstretch{1.75}
	\textbf{\LARGE AgroSíntese:}\\[0.5em]
	\textbf{\LARGE Uma Plataforma Dialógica e Inclusiva de ATER via WhatsApp com Inteligência Artificial e Regionalismos Linguísticos}\\[0.5em]
\end{center}

\vspace{2em}

\begin{center}
	\setstretch{1}
	\textbf{Jonas Ernesto Poli}\\
	Mestrando em Ciência, Tecnologia e Sociedade pela Universidade\\
	Federal de São Carlos, Brasil.\\
	E-mail: \href{mailto:jonaspoli@ufscar.br}{jonaspoli@ufscar.br}
\end{center}

\vspace{2em}

\begin{center}
	\setstretch{1}
	\textbf{Luís Fernando Soares Zuin}\\
    Doutor em Engenharia de Produção\\
	Federal de São Carlos, Brasil. Professora da Universidade Federal de\\
	São Carlos, Brasil.\\
	E-mail: \href{mailto:zuin@ufscar.br}{zuin@ufscar.br}
\end{center}
	
	
	%------------------------------------------------------------------------------------%
	% R E S U M O   N O   I D I O M A   D O   T E X T O
	%
	% OBRIGATÓRIO. Elaborado conforme a NBR 6028. 
	% Deve ser redigido em um só parágrafo contendo de 150 a 500 palavras e ressaltar: objetivo, método, resultados e as principais conclusões.
	% Após o resumo, são listadas palavras-chave relacionadas à temática do trabalho, separadas entre si por ponto e virgula (;) e finalizadas por ponto final.  (NBR 6028)
	%------------------------------------------------------------------------------------%
	
	\begin{resumo}
		\setstretch{1}
		Este artigo apresenta o AgroSíntese, um sistema inovador de Assistência Técnica e Extensão Rural (ATER), baseado em inteligência artificial (IA) e integrado ao WhatsApp, que busca atender às necessidades específicas dos agricultores familiares brasileiros por meio de uma comunicação dialógica e culturalmente situada. O AgroSíntese utiliza o modelo de IA GPT4All, treinado com conteúdos técnicos agrícolas específicos e regionalizados, permitindo que as respostas geradas sejam personalizadas em texto ou áudio, com síntese de voz adaptada aos sotaques e expressões regionais brasileiras. A fundamentação teórica deste trabalho baseia-se principalmente nas contribuições de Mikhail Bakhtin sobre o dialogismo e o caráter social da linguagem, na abordagem educativa crítica e dialógica de Paulo Freire, na visão da experiência como elemento transformador proposta por Jorge Larrosa e nas perspectivas sobre comunicação rural e digitalização da extensão rural desenvolvidas por Luís Fernando Soares Zuin e Renato Lopes. O AgroSíntese visa democratizar e facilitar o acesso ao conhecimento técnico agrícola, aumentando significativamente a inclusão digital, a compreensão e aplicação prática das técnicas agrícolas pelos agricultores. Além disso, espera-se promover o fortalecimento das identidades culturais locais e da autonomia produtiva e social das comunidades rurais, alinhando-se aos Objetivos de Desenvolvimento Sustentável (ODS) da ONU. Em termos práticos, a utilização dessa tecnologia prevê impactos positivos diretos na produtividade agrícola, na sustentabilidade econômica e na valorização das diversidades linguísticas e culturais do Brasil rural.
		
		\vspace*{0.5cm}
		
		\noindent\textbf{{Palavras-Chave: }} Assistência Técnica e Extensão Rural; Inteligência Artificial; GPT4All; Comunicação Dialógica; Agricultura Familiar; WhatsApp; ATER Digital Participativa.
		
	\end{resumo}
	
	%------------------------------------------------------------------------------------%
	% A B S T R A C T :   R E S U M O   N O   I D I O M A   E S T R A N G E I R O
	%
	% OBRIGATÓRIO. Elaborado com as mesmas características do resumo em língua portuguesa.
	% Se redigido em inglês-ABSTRACT, em castelhano-RESUMEN, em francês-RÉSUMÉ.
	% Após o resumo, são listadas palavras-chave relacionadas à temática do trabalho no 
	% idioma escolhido. Se redigido em inglês - KEYWORDS, em espanhol - PALABRAS CLAVES,
	% em francês - MOTS-CLÉS.
	% TEXTO ATUAL 
	%ZUSAMMENFASSUNG (alemão)
	%RIASSUNTO (italiano)
	
	%Keywords (inglês)
	%Palabras clave (espanhol)
	%Mot-clé (francês)
	%Stichwörter (alemão)
	%Parole chiave (italiano)
	%------------------------------------------------------------------------------------%
	\begin{center}
		\setstretch{1}
		\textbf{\large AgroSynthesis: A Dialogical and Inclusive ATER Platform via WhatsApp with Artificial Intelligence and Linguistic Regionalisms}
	\end{center}	
	\begin{resumo}[Abstract] % Substitua 'Abstract' pela palavra no idioma desejado, caso precise.
		\setstretch{1}
		This paper presents AgroSíntese, an innovative Technical Assistance and Rural Extension (ATER) system based on artificial intelligence (AI) integrated with WhatsApp, designed specifically to address the unique communication and technical needs of Brazilian family farmers. AgroSíntese employs the GPT4All AI model, specially trained with regionally adapted agricultural content, enabling personalized textual and audio responses through speech synthesis technology attuned to the regional accents and linguistic nuances of Brazil. The theoretical foundation of this study draws primarily from Mikhail Bakhtin's concepts of dialogism and the inherently social nature of language, Paulo Freire's critical and dialogical educational approach, Jorge Larrosa’s emphasis on experience as transformative, and Luís Fernando Soares Zuin and Renato Lopes’ work on rural communication and digital extension methodologies. AgroSíntese seeks to democratize access to specialized agricultural knowledge, significantly enhancing digital inclusion and ensuring clearer understanding and practical implementation of farming techniques. Furthermore, it aims to strengthen local cultural identities and the productive and social autonomy of rural communities, contributing to the United Nations' Sustainable Development Goals (SDGs). Practically, the deployment of this technology is expected to yield direct positive impacts on agricultural productivity, economic sustainability, and the recognition and preservation of Brazil’s diverse linguistic and cultural landscapes in rural settings.
		
		\vspace*{0.5cm}
		
		%Subistitua 'Keywords' pela palavra no idioma desejado, caso precise.
		\noindent\textbf{{Keywords: }} Technical Assistance and Rural Extension; Artificial Intelligence; GPT4All; Dialogical Communication; Family Farming; WhatsApp; Participatory Digital Extension.
		
		
	\end{resumo}
	

	
	

	
