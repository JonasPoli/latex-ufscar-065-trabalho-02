
%----------------------------------------------------------------------------------------%
% I N Í C I O   D O   D O C U M E N T O - P R É   T E X T U A L

%----------------------------------------------------------------------------------------%
\begin{document}
	
	\imprimircapa
	\imprimirfolhaderosto
	
	%------------------------------------------------------------------------------------%
	%ELEMENTO OPCIONAL. Consulte a seção de pós-graduação da sua unidade e verifique a obrigatoriedade ou não deste item. Se não ouver necessidade, exclua essa página. 
	
	\begin{resumo}[IMPACTO POTENCIAL DESTA PESQUISA]
		O impacto esperado na sociedade deve ser redigido de forma sucinta considerando, os seguintes aspectos: o potencial científico, técnico, social, inovador, econômico, educacional e cultural; a internacionalização; a inserção local, regional e nacional;  o desenvolvimento sustentável, conhecimento e temática da pesquisa. 
		
		\vspace*{0.5cm}
		
		%Subistitua pela palavra no idioma desejado, caso precise.
		\begin{center}
			\noindent\textbf{\MakeUppercase{POTENTIAL IMPACT OF THIS RESEARCH }}
		\end{center}
		
		
		\vspace*{0.5cm}
		
		The expected impact on society should be written succinctly, considering the following aspects: the scientific, technical, social, innovative, economic, educational and cultural potential; internationalization; local, regional and national insertion; sustainable development, knowledge and research themes.
		
	\end{resumo}
	
	% F O L H A   D E   A P R O V A Ç Ã O
	%
	% OBRIGATÓRIO. Este é o modelo de folha de aprovação que deve ser digitalizada após as assinaturas da banca. Utilize um software de edição de PDF para substituição posterior dessa folha. Se possível, utilize um recurso de assinatura digital fornecido pela ferramenta de edição de PDF, evitando assim a digitalização da folha toda.
	%------------------------------------------------------------------------------------%
	\begin{folhadeaprovacao}
		
		\begin{center}
			\normalsize{\textbf{\MakeUppercase\nomeDoAutor}}
			\\
			\vspace*{1cm}
			\normalsize{\textbf{\MakeUppercase\tituloDoTrabalho:}}
			\\
			\subtituloDoTrabalho
		\end{center}
		
		\vspace*{2cm}
		
		\noindent\tipoDeTrabalho\ apresentado(a) à Universidade Estadual Paulista (UNESP),\ \NomeDaFaculdade,\ \nomeDaCidade, para obtenção do título de\ \GrauAcademico{ }em\ \nomeDoCurso.
		\par
		\vspace*{1cm} 
		\noindent {Área de Concentração: \AreaDeConcentracao}
		
		\vspace*{1cm} 
		
		\noindent Data de defesa: \dataDeDefesa
		\vspace*{1cm}
		
		\noindent{BANCA EXAMINADORA}
		
		\vspace*{1cm}
		
		\raggedright\makebox[2.5in]{\hrulefill}\\
		\tituloDoOrientador \nomeDoOrientador \\ UNESP -- \NomeDaFaculdade\ -- Campus de \nomeDaCidade
		
		\vspace*{1cm}
		
		\raggedright\makebox[2.5in]{\hrulefill}\\
		\tituloDoMembroA\ \nomeDoMembroA \\ \instituicaomembroA
		
		\vspace*{1cm}
		
		\raggedright\makebox[2.5in]{\hrulefill}\\
		\tituloDoMembroB\ \nomeDoMembroB \par \instituicaomembroB
		
		\vspace*{1cm}
		
		\raggedright\makebox[2.5in]{\hrulefill}\\
		\tituloDoMembroC\ \nomeDoMembroC \\ \instituicaomembroC
		
		\vspace*{1cm}
		
		\raggedright\makebox[2.5in]{\hrulefill}\\
		\tituloDoMembroD\ \nomeDoMembroD \\ \instituicaomembroD
		
	\end{folhadeaprovacao}
	
	
	%------------------------------------------------------------------------------------%
	% D E D I C A T Ó R I A
	%
	% ELEMENTO OPCIONAL. Se não for utilizar uma dedicatória, basta apagar todo o código dessa seção.
	%------------------------------------------------------------------------------------%
	
	\begin{dedicatoria}
		\vspace*{\fill}
		\begin{flushright}
			Dedico este trabalho à … 
		\end{flushright}
		\vspace*{1cm}
	\end{dedicatoria}
	
	%------------------------------------------------------------------------------------%
	% A G R A D E C I M E N T O S
	%
	% ELEMENTO OPCIONAL. PORÉM É OBRIGATÓRIO PARA BOLSISTAS. 
	% Citar as pessoas, instituição, agência de fomento, entre outros que contribuíram de maneira relevante à elaboração do trabalho e na vida acadêmica.
	% Se não for utilizar os agradecimentos, basta apagar todo o código dessa seção.
	%------------------------------------------------------------------------------------%
	\begin{agradecimentos}
		
		É um texto em que o autor agradece as pessoas que contribuíram de forma relevante para o desenvolvimento do trabalho, quando houver apoio financeiro à pesquisa, deve-se obrigatoriamente mencionar esse apoio nos agradecimentos, conforme o que prevê cada agência.
		
		O presente trabalho foi realizado com apoio da Coordenação de Aperfeiçoamento de Pessoal de Nível Superior – Brasil (CAPES) – Código de Financiamento 001.
		
		À FAPESP, pelo apoio financeiro, concedido por meio do Processo nº aaaa/nnnnn-d, Fundação de Amparo à Pesquisa do Estado de São Paulo (FAPESP).
		
		Ao Conselho Nacional de Desenvolvimento Científico e Tecnológico (CNPq), pela concessão de bolsa de pesquisa (processo XXX).
		
		
	\end{agradecimentos}
	
	%------------------------------------------------------------------------------------%
	
	%------------------------------------------------------------------------------------%
	% E P Í G R A F E
	%
	% ELEMENTO OPCIONAL. Texto em que o autor apresenta uma citação, seguida de indicação de autoria.
	% Se não for utilizar uma epígrafe, basta apagar todo o código dessa seção.
	%------------------------------------------------------------------------------------%
	
	\begin{epigrafe}
		\vspace*{\fill}
		\begin{flushright}
			\textit{
				``As universidades serão o que são suas bibliotecas``\\ 
				\cite[~ p. 19, tradução nossa]{Gelfand}
			}
		\end{flushright}
	\end{epigrafe}
	
	%------------------------------------------------------------------------------------%
	% R E S U M O   N O   I D I O M A   D O   T E X T O
	%
	% OBRIGATÓRIO. Elaborado conforme a NBR 6028. 
	% Deve ser redigido em um só parágrafo contendo de 150 a 500 palavras e ressaltar: objetivo, método, resultados e as principais conclusões.
	% Após o resumo, são listadas palavras-chave relacionadas à temática do trabalho, separadas entre si por ponto e virgula (;) e finalizadas por ponto final.  (NBR 6028)
	%------------------------------------------------------------------------------------%
	
	\begin{resumo}
		
		O texto deve ser composto por frases concisas em parágrafo único, utilizando o verbo na terceira pessoa. Evite símbolos e contrações que não sejam de uso corrente. Evite fórmulas, equações, diagramas etc., que não sejam absolutamente necessários; quando seu emprego for imprescindível, defini-los na primeira vez que aparecerem. Um resumo deve conter entre 150 a 500 palavras. As palavras-chave devem ficar logo abaixo do resumo, separadas entre si por ponto e vírgula e finalizadas por ponto. Além disso, devem ser grafadas com as iniciais em letra minúscula, exceto nomes próprios e nomes científicos. Para a escolha das palavras-chave consulte o Tesauro Unesp ou descritores autorizados da área, que representam o conteúdo do trabalho.
		
		\vspace*{0.5cm}
		
		\noindent\textbf{{Palavras-Chave: }} palavra-chave; palavra-chave; palavra-chave.
		
	\end{resumo}
	
	%------------------------------------------------------------------------------------%
	% A B S T R A C T :   R E S U M O   N O   I D I O M A   E S T R A N G E I R O
	%
	% OBRIGATÓRIO. Elaborado com as mesmas características do resumo em língua portuguesa.
	% Se redigido em inglês-ABSTRACT, em castelhano-RESUMEN, em francês-RÉSUMÉ.
	% Após o resumo, são listadas palavras-chave relacionadas à temática do trabalho no 
	% idioma escolhido. Se redigido em inglês - KEYWORDS, em espanhol - PALABRAS CLAVES,
	% em francês - MOTS-CLÉS.
	% TEXTO ATUAL 
	%ZUSAMMENFASSUNG (alemão)
	%RIASSUNTO (italiano)
	
	%Keywords (inglês)
	%Palabras clave (espanhol)
	%Mot-clé (francês)
	%Stichwörter (alemão)
	%Parole chiave (italiano)
	%------------------------------------------------------------------------------------%
	
	\begin{resumo}[Abstract] % Substitua 'Abstract' pela palavra no idioma desejado, caso precise.
		
		The text should consist of concise sentences in a single paragraph, using the third person singular form of the verb. Avoid symbols and contractions that are not commonly used. Avoid formulas, equations, diagrams, etc., unless absolutely necessary; when their use is essential, define them the first time they appear. An abstract should contain between 150 to 500 words. Keywords should be placed right below the abstract, separated by semicolons and ending with a period. Additionally, they should be written in lowercase except for proper names and scientific names. To choose keywords, consult the Tesauro Unesp or authorized descriptors in the field that represent the content of the work.
		
		\vspace*{0.5cm}
		
		%Subistitua 'Keywords' pela palavra no idioma desejado, caso precise.
		\noindent\textbf{{Keywords: }} keyword; keyword; keyword.
		
	\end{resumo}
	
	%------------------------------------------------------------------------------------%
	% L I S T A   D E   I L U S T R A Ç Õ E S
	%
	% ELEMENTO OPCIONAL. Deve ser elaborada de acordo com a ordem em que se apresenta no texto, podendo ser em lista própria para cada tipo de ilustração ou lista única para variados tipos de ilustrações (figuras, fotografias, organogramas, quadros, etc).(NBR: 14724)
	% Se não for utilizar uma Lista de Ilustrações, basta apagar todo o código dessa seção.
	
	
	%------------------------------------------------------------------------------------%
	
	\listoffigures*
	\newpage
	
	%------------------------------------------------------------------------------------%
	% L I S T A   D E   T A B E L A S
	%
	% ELEMENTO OPCIONAL. Deve ser elaborada de acordo com a ordem em que se apresenta no texto.
	%Se não for utilizar uma Lista de Tabelas, basta apagar todo o código dessa seção.
	%------------------------------------------------------------------------------------%
	
	\listoftables*
	\newpage
	
	%------------------------------------------------------------------------------------%
	% L I S T A   D E   A B R E V I A T U R A S   E   S I G L A S
	%
	% ELEMENTO OPCIONAL. Deve ser elaborada em ordem alfabética, seguidas das palavras ou expressões correspondentes por extenso.
	% Se não for utilizar uma Lista de Abreviaturas, basta apagar todo o código dessa seção.
	
	%------------------------------------------------------------------------------------%
	
	\begin{siglas}
		\item [ACT]	Administração Científica do Trabalho 
		\item [AIT]	Associação Internacional do Trabalho 
		\item [CAPES]	Coordenação de Aperfeiçoamento de Pessoal de Nível Superior 
		\item [IES]	Instituição de Ensino Superior 
		\item [MEC]	Ministério da Educação 
		\item [OMS]	Organização Mundial da Saúde
	\end{siglas}  
	
	
	%------------------------------------------------------------------------------------%
	% L I S T A   D E   S Í M B O L O S
	%
	% ELEMENTO OPCIONAL. Deve ser elaborada de acordo com a ordem em que se apresenta no texto, acompanhado com seus respectivos significados. Se não for utilizar uma Lista de Símbolos, basta apagar todo o código dessa seção.
	
	%------------------------------------------------------------------------------------%
	
	\begin{simbolos}
		\item[$\alpha$] Letra Grega Alfa
		\item[$\beta$] Letra grega Beta
		\item[$\gamma$] Letra grega Gama
		\item[$e$] Número de Euler
		\item[R\$] Unidade monetária Brasileira (Real)
	\end{simbolos}
	
	%------------------------------------------------------------------------------------%
	% S U M Á R I O
	%
	% OBRIGATÓRIO. Recomenda-se atualizar o sumário apenas no fim da elaboração do trabalho. 
	% Havendo mais de um volume, cada um deve conter o sumário completo do trabalho (NBR: 6024; NBR: 6027). 
	% Não se deve confundir sumário com índice.
	
	%------------------------------------------------------------------------------------%
	
	\tableofcontents
	
