
%----------------------------------------------------------------------------------------%
% I N Í C I O   D O   D O C U M E N T O - P R É   T E X T U A L

%----------------------------------------------------------------------------------------%






\begin{center}
	\setstretch{1.75}
	\textbf{\LARGE Cartografia dos Sotaques Brasileiros:}\\[0.5em]
	\textbf{\LARGE Uma proposta de identificação e organização de falantes por código DDD}\\[0.5em]
\end{center}

\vspace{2em}

\begin{center}
	\setstretch{1}
	\textbf{Jonas Ernesto Poli}\\
	Mestrando em Ciência, Tecnologia e Sociedade pela Universidade\\
	Federal de São Carlos, Brasil.\\
	E-mail: \href{mailto:jonaspoli@ufscar.br}{jonaspoli@ufscar.br}
\end{center}

\vspace{2em}

\begin{center}
	\setstretch{1}
	\textbf{Luís Fernando Soares Zuin}\\
    Doutor em Engenharia de Produção\\
	Federal de São Carlos, Brasil. Professora da Universidade Federal de\\
	São Carlos, Brasil.\\
	E-mail: \href{mailto:zuin@ufscar.br}{zuin@ufscar.br}
\end{center}
	
	
	%------------------------------------------------------------------------------------%
	% R E S U M O   N O   I D I O M A   D O   T E X T O
	%
	% OBRIGATÓRIO. Elaborado conforme a NBR 6028. 
	% Deve ser redigido em um só parágrafo contendo de 150 a 500 palavras e ressaltar: objetivo, método, resultados e as principais conclusões.
	% Após o resumo, são listadas palavras-chave relacionadas à temática do trabalho, separadas entre si por ponto e virgula (;) e finalizadas por ponto final.  (NBR 6028)
	%------------------------------------------------------------------------------------%
	
	\begin{resumo}
		\setstretch{1}

 Este artigo propõe a utilização do mapa de Discagem Direta à Distância (DDD) como forma para a identificação dos dialetos brasileiros. O objetivo central é desenvolver um subsídio para tecnologias assistivas que se adaptem às diversas manifestações da fala regional. A pesquisa parte da divisão de dialeto de Antenor Nascentes, de 1922, e avança até estudos contemporâneos do Atlas Linguístico do Brasil (ALiB), estabelecendo uma correlação entre os códigos de área telefônicos e os dialetos identificados no país. O embasamento teórico ancora-se nas concepções de dialogismo de Bakhtin e na pedagogia de Paulo Freire, que compreendem a linguagem como fenômeno social e a comunicação como intercâmbio. A metodologia consistiu na sobreposição do mapa dialetal com os códigos DDD, identificando convergências e lacunas. Os resultados indicam que, apesar de certas limitações, o sistema DDD pode ser um recurso valioso na criação de tecnologias que respeitem a diversidade linguística brasileira, com especial atenção às áreas rurais e à Assistência Técnica e Extensão Rural (ATER) digital participativa. Conclui-se que esta abordagem representa um passo significativo para o desenvolvimento de ferramentas tecnológicas que reconheçam e valorizem os sotaques locais, proporcionem a inclusão digital e o respeito às identidades culturais. 
\\[0.5em]

		
		\vspace*{0.5cm}
		
		\noindent\textbf{{Palavras-Chave: }}  Dialetos brasileiros; Códigos DDD; Tecnologias assistivas; Inclusão digital; Variação linguística.

		
	\end{resumo}
	
	%------------------------------------------------------------------------------------%
	% A B S T R A C T :   R E S U M O   N O   I D I O M A   E S T R A N G E I R O
	%
	% OBRIGATÓRIO. Elaborado com as mesmas características do resumo em língua portuguesa.
	% Se redigido em inglês-ABSTRACT, em castelhano-RESUMEN, em francês-RÉSUMÉ.
	% Após o resumo, são listadas palavras-chave relacionadas à temática do trabalho no 
	% idioma escolhido. Se redigido em inglês - KEYWORDS, em espanhol - PALABRAS CLAVES,
	% em francês - MOTS-CLÉS.
	% TEXTO ATUAL 
	%ZUSAMMENFASSUNG (alemão)
	%RIASSUNTO (italiano)
	
	%Keywords (inglês)
	%Palabras clave (espanhol)
	%Mot-clé (francês)
	%Stichwörter (alemão)
	%Parole chiave (italiano)
	%------------------------------------------------------------------------------------%
	\begin{center}
		\setstretch{1}
		\textbf{\large Cartography of Brazilian Accents: a proposal for identifying and organising speakers by DDD area code}
	\end{center}	
	\begin{resumo}[Abstract] % Substitua 'Abstract' pela palavra no idioma desejado, caso precise.
		\setstretch{1}
This article proposes using the Direct Distance Dialing (DDD) area code map as a foundation for identifying Brazilian dialects. The main goal is to develop material for assistive technologies that adapt to diverse regional speech patterns. The research begins with Antenor Nascentes' 1922 dialect division and extends to contemporary studies from the Linguistic Atlas of Brazil (ALiB), correlating telephone area codes with identified dialects in the country. The theoretical basis is rooted in Bakhtin's concepts of dialogism and Paulo Freire's pedagogy, which view language as a social phenomenon and communication as an exchange. The methodology involved overlaying the dialect map with DDD codes, identifying convergences and gaps. The results indicate that, despite certain limitations, the DDD system can be a valuable resource for creating technologies that respect Brazil's linguistic diversity, particularly in rural areas and for participatory digital Technical Assistance and Rural Extension (ATER). It is concluded that this approach represents a significant step towards developing technological tools that recognize and value local accents, promoting digital inclusion and respect for cultural identities.
\\[0.5em]

		
		\vspace*{0.5cm}
		
		%Subistitua 'Keywords' pela palavra no idioma desejado, caso precise.
		\noindent\textbf{{Keywords: }}
        Brazilian dialects; DDD area codes; Assistive technologies; Digital inclusion; Linguistic variation.
		
		
	\end{resumo}
	

	
	

	
