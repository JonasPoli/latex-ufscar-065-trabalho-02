
%----------------------------------------------------------------------------------------%
% I N Í C I O   D O   D O C U M E N T O - P R É   T E X T U A L

%----------------------------------------------------------------------------------------%






\begin{center}
	\setstretch{1.75}
	\textbf{\LARGE Cartografia dos Sotaques Brasileiros:}\\[0.5em]
	\textbf{\LARGE Uma proposta de identificação e organização de falantes por código DDD}\\[0.5em]
\end{center}

\vspace{2em}

\begin{center}
	\setstretch{1}
	\textbf{Jonas Ernesto Poli}\\
	Mestrando em Ciência, Tecnologia e Sociedade pela Universidade\\
	Federal de São Carlos, Brasil.\\
	E-mail: \href{mailto:jonaspoli@ufscar.br}{jonaspoli@ufscar.br}
\end{center}

\vspace{2em}

\begin{center}
	\setstretch{1}
	\textbf{Luís Fernando Soares Zuin}\\
    Doutor em Engenharia de Produção\\
	Federal de São Carlos, Brasil. Professora da Universidade Federal de\\
	São Carlos, Brasil.\\
	E-mail: \href{mailto:zuin@ufscar.br}{zuin@ufscar.br}
\end{center}
	
	
	%------------------------------------------------------------------------------------%
	% R E S U M O   N O   I D I O M A   D O   T E X T O
	%
	% OBRIGATÓRIO. Elaborado conforme a NBR 6028. 
	% Deve ser redigido em um só parágrafo contendo de 150 a 500 palavras e ressaltar: objetivo, método, resultados e as principais conclusões.
	% Após o resumo, são listadas palavras-chave relacionadas à temática do trabalho, separadas entre si por ponto e virgula (;) e finalizadas por ponto final.  (NBR 6028)
	%------------------------------------------------------------------------------------%
	
	\begin{resumo}
		\setstretch{1}

Este artigo sugere usar o mapa de Discagem Direta à Distância (DDD) como base para identificar os dialetos brasileiros. O objetivo é preparar um material para ser utilizado por tecnologias assistivas que se ajustem às diferentes formas de falar em várias regiões. A pesquisa se inicia na divisão de dialetos feita por Antenor Nascentes em 1922 e vai até estudos mais recentes do Atlas Linguístico do Brasil (ALiB), ligando os códigos de área telefônicos aos dialetos encontrados no Brasil. A teoria se apoia nas ideias de Bakhtin sobre diálogo e na pedagogia de Paulo Freire, entendendo a linguagem como algo que acontece na sociedade e a comunicação como troca. A metodologia incluiu sobrepor o mapa dos dialetos com os DDDs, identificando semelhanças e lacunas. Os resultados mostram que, apesar de algumas limitações, o sistema de DDD pode ajudar a criar tecnologias que respeitem a diversidade linguística do Brasil, principalmente em áreas rurais e na ATER digital participativa. Concluímos que essa abordagem é um bom passo para desenvolver ferramentas tecnológicas que reconheçam e valorizem os sotaques locais, promovendo inclusão digital e respeito às identidades culturais.
\\[0.5em]

		
		\vspace*{0.5cm}
		
		\noindent\textbf{{Palavras-Chave: }}  Dialetos brasileiros; DDD; Dialogismo; Tecnologias assistivas; Comunicação rural.

		
	\end{resumo}
	
	%------------------------------------------------------------------------------------%
	% A B S T R A C T :   R E S U M O   N O   I D I O M A   E S T R A N G E I R O
	%
	% OBRIGATÓRIO. Elaborado com as mesmas características do resumo em língua portuguesa.
	% Se redigido em inglês-ABSTRACT, em castelhano-RESUMEN, em francês-RÉSUMÉ.
	% Após o resumo, são listadas palavras-chave relacionadas à temática do trabalho no 
	% idioma escolhido. Se redigido em inglês - KEYWORDS, em espanhol - PALABRAS CLAVES,
	% em francês - MOTS-CLÉS.
	% TEXTO ATUAL 
	%ZUSAMMENFASSUNG (alemão)
	%RIASSUNTO (italiano)
	
	%Keywords (inglês)
	%Palabras clave (espanhol)
	%Mot-clé (francês)
	%Stichwörter (alemão)
	%Parole chiave (italiano)
	%------------------------------------------------------------------------------------%
	\begin{center}
		\setstretch{1}
		\textbf{\large Cartography of Brazilian Accents: a proposal for identifying and organising speakers by DDD area code}
	\end{center}	
	\begin{resumo}[Abstract] % Substitua 'Abstract' pela palavra no idioma desejado, caso precise.
		\setstretch{1}
This article proposes using the Direct Distance Dialing (DDD) area code map as a foundation for identifying Brazilian dialects. The goal is to prepare material to be used by assistive technologies that adapt to the different ways of speaking found across various regions. The research begins with the dialect division proposed by Antenor Nascentes in 1922 and extends to more recent studies from the Linguistic Atlas of Brazil (ALiB), linking telephone area codes to the dialects observed throughout the country. The theory is grounded in Bakhtin’s ideas on dialogue and Paulo Freire’s pedagogy, viewing language as a social phenomenon and communication as an exchange. The methodology involved overlaying the dialect map with the DDD codes, identifying both similarities and gaps. The results show that, despite some limitations, the DDD system can help create technologies that respect Brazil’s linguistic diversity, especially in rural areas and within participatory digital ATER. We conclude that this approach represents a meaningful step toward developing technological tools that recognize and value local accents, promoting digital inclusion and respect for cultural identities.
\\[0.5em]

		
		\vspace*{0.5cm}
		
		%Subistitua 'Keywords' pela palavra no idioma desejado, caso precise.
		\noindent\textbf{{Keywords: }} Brazilian dialects; DDD; Dialogism; Assistive technologies; Rural communication.
		
		
	\end{resumo}
	

	
	

	
