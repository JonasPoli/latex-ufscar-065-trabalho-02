\chapter{REGRAS DE APRESENTAÇÃO} 
% % Nesse capítulo será ilustrado como se utilizar os diferentes níveis e subníveis de seção em seu documento. 
%Fique atento quanto a identação do código para manter a clareza e organização de seu trabalho.
% ATENÇÃO para todo título de segunda seção (nivel 2) deve ser escrita OBRIGATORIAMENTE com letras maiúsculas. Esse padrão deve ser mantido em todo texto que irá ajudar no sumário.

Essa seção traz vários exemplos para a redação do trabalho acadêmico confome a NBR 14724 \cite{NBR14724}. Após a leitura, apague seu conteúdo ou reutilize as seções para a elaboração do seu trabalho acadêmico.

\section{ALÍNEAS E SUBALÍNEAS} 

De acordo com Norma Brasileira 6024, da Associação Brasileira de Normas Técnica \cite{NBR6024}, caso seja necessário elencar assuntos que não possuam títulos dentro de uma mesma seção, estes podem ser subdivididos em alíneas e subalíneas:

\begin{enumerate}[label=\alph*)]
	\item exemplo de uma alínea: o texto que vem dentro da alínea é separado por ponto e vírgula;
	\item exemplo de uma segunda alínea: o texto que finaliza a alínea termina em ponto final:
	\begin{itemize}
		\item[--] exemplo de subalínea.
	\end{itemize}
\end{enumerate}       

\section{CITAÇÕES}

Esse capítulo apresenta regras e exemplos para a elaboração de citações. As citações devem ser apresentadas conforme a NBR 10520 \cite{NBR10520}, e devem ser usadas em conjunto com a norma de referências, a \citeonline{NBR6023}.
Nesta seção serão apresentados alguns exemplos de citações mais utilizadas. Para os demais tipos, consulte o e-book: Manual de normalização de trabalhos acadêmicos: citação e referência: ABNT \cite{unesp2020ref}.


\subsection{Citações diretas}

Citação direta é a transcrição textual de parte da obra do autor consultado. A indicação de paginação ou localização é obrigatória.

\subsubsection{Citação direta com menos de quatro linhas}

Indicar o trecho com aspas duplas. Exemplo:
Segundo \citeonline[~ p.276]{Rouanet}, “[...] não somos humanos fora da cultura, mas não seremos homens livres se não pudermos, sempre que necessário, assumir uma posição de exterioridade com relação ao mundo social”.

\subsubsubsection{Citação direta com quatro ou mais linhas}	

Indicar com fonte tamanho 10, espaçamento simples e sem aspas. Recomenda-se o recuo de 4 cm da margem esquerda. Exemplo:
% Utilize esse código toda a vez que for fazer uma citação direta com mais de 3 linhas. Mantenha todos os valores de alinhamento conforme o exemplo, substitua somente o conteúdo do texto pelo texto que irá utilizar.
\vspace{1.5pt}
\begin{flushright}
	\begin{minipage}{.724\textwidth}
		{\SingleSpacing\small
			
			Mas não há ninguém tão assustado, ou que tenha medos tão estranhos, quanto os conquistadores. Eles evocam intermináveis fantasmas, aterrorizados com a ideia de que suas vítimas um dia façam com eles o que fizeram com elas... mesmo que, na verdade, as vítimas não deem a mínima para essa mesquinharia e tenham seguido em frente \cite[~ p.253]{Jemisin}
		}
	\end{minipage}
\end{flushright}
\vspace{1.5pt}


\subsection{Citações indiretas}	

Citação indireta é o texto baseado na obra do autor consultado. A indicação da página ou localização é opcional. Exemplo:

Segundo Daniel Munduruku \cite{aruju}, cabe aos adultos oferecer condições às crianças para que elas se desenvolvam plenamente para que possam se tornar humanos completos. Assim, segundo o educador, é preciso educar as crianças para o agora, para o presente, sem pular as etapas da vida, pois pensar o futuro é renunciar o presente.

\subsection{Citação de citação}	

Define-se como texto citado por outro autor dentro do documento que está sendo consultado, do qual não se teve acesso ao original. Exemplo:
\\
Segundo Freire (1994, p. 13, \textit{apud} \cite[p.17]{streck}, “[...] a pedagogia do oprimido como centro, me aparecem tão atuais quanto outros a que me refiro dos anos 80 e de hoje”.

% Para citação de citação descreva o sobrenome do autor da obra que não teve acesso,´entre parenteses indique o ano e página se houver e acrescente a expressão apud e a citação do autor que tem acesso inclusive a página) 

