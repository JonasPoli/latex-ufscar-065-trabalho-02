    
%----------------------------------------------------------------------------------------%
% I N F O R M A Ç Õ E S   B Á S I C A S   S O B R E   O   T R A B A L H O
%
% Defina aqui as informações pertinentes ao trabalho.
%----------------------------------------------------------------------------------------%

%----------------------------------------------------------------------------------------%
% D A D O S   P E S S O A I S
%----------------------------------------------------------------------------------------%

%Nome completo do autor do presente Trabalho acadêmico:
\newcommand{\nomeDoAutor}{
	Jonas Ernesto Poli
}

%Nome do curso:
\newcommand{\nomeDoCurso}{PPGCTS}

%----------------------------------------------------------------------------------------%
% D A D O S   S O B R E   O   T R A B A L H O
%----------------------------------------------------------------------------------------%

%Título do presente trabalho:
\newcommand{\tituloDoTrabalho}{
	AgroSíntese: Um Sistema de Assistência Técnica e Extensão Rural com Inteligência Artificial para Atendimento ao Produtor Rural via WhatsApp
}

%Subtítulo do presente trabalho, se houver:
\newcommand{\subtituloDoTrabalho}{
	%subtítulo se houver
}

%Tipo do trabalho (Dissertação / Tese / Trabalho de Conclusão de Curso / Trabalho de Conclusão de Residencia) 
\newcommand{\tipoDeTrabalho}{Artigo}

% Nome da Faculdade ou instituto
\newcommand{\NomeDaFaculdade}{UFSCAR}

%Grau acadêmico (Mestre(a) / Doutor(a) / Bachareal / Licenciatura etc)
\newcommand{\GrauAcademico}{Mestre}

% área de concentração, se houver (muito utilizado no mestrado e doutorado) % remover no pacote básico se não houver 
\newcommand{\AreaDeConcentracao}{
	Ciência, Tecnologia e Sociedade
}

%Mês da entrega do trabalho 
\newcommand{\dataDeDefesa}{
	99/99/9999
}

%Ano da entrega do trabalho
\newcommand{\anoDeEntrega}{
	2025
}

%----------------------------------------------------------------------------------------%
% D A D O S   D O S   O R I E N T A D O R E S,   B A N C A   E   C O O R D E N A D O R
%----------------------------------------------------------------------------------------%


%Nome do professor
\newcommand{\nomeDoProfessor}{
	Luís Fernando Soares Zuin
}
% acrescentar intituição 
%Título do professor:
\newcommand{\tituloDoProfessor}{
	Prof. Dr.
}


%Nome do orientador do presente trabalho:
\newcommand{\nomeDoOrientador}{
	Professora Dra. Maria Teresa Miceli Kerbauy
}
% acrescentar intituição 
%Título do orientador:
\newcommand{\tituloDoOrientador}{
	Profª Dra.
}

%Nome do coorientador, se houver:
\newcommand{\NomeDoCoorientador}{
	Nome Completo do Coorientador
}

%Título do coorientador:
\newcommand{\tituloDoCoorientador}{
	Profº Dr.
}

%instituição do orientador 
\newcommand{\instituicaoorientador}{PPGCTS/UFSCar – São Carlos/SP}     

%instituição do coorientador 
\newcommand{\instituicaocoorientador}{PPGCTS/UFSCar – São Carlos/SP}

%Nome do Membro da Banca 1 :
\newcommand{\nomeDoMembroA}{Nome Completo do Membro da Banca}

%Título do Membro da Banca 1:
\newcommand{\tituloDoMembroA}{Profº Dr.}
%instituição do Membro da Banca 1 
\newcommand{\instituicaomembroA}{Nome da Instituição do membro 1}

%Nome do membro da Banca 2:
\newcommand{\nomeDoMembroB}{Nome Completo do Membro banca}

%Título do Membro da Banca 2:
\newcommand{\tituloDoMembroB}{Profº Dr.}
% Instituição do Membro 2
\newcommand{\instituicaomembroB}{Nome da Instituição do membro 2}

%ATENÇÃO! Geralmente Tese de Doutorado possui mais de 2 membros na banca, então é possível remover no pacote básico. 
%Nome do membro da Banca 3:
\newcommand{\nomeDoMembroC}{Nome Completo do Membro banca}

%Título do Membro da Banca 3:
\newcommand{\tituloDoMembroC}{Profº Dr.}
% Instituição do Membro 3
\newcommand{\instituicaomembroC}{Nome da Instituição do membro 3}
%Nome do membro da Banca 4:
\newcommand{\nomeDoMembroD}{Nome Completo do Membro banca}

%Título do Membro da Banca 4:
\newcommand{\tituloDoMembroD}{Profº Dr.}
% Instituição do Membro 4
\newcommand{\instituicaomembroD}{Nome da Instituição do membro 4}
%----------------------------------------------------------------------------------------%
% D A D O S   D A   I N S T I T U I Ç Ã O
%----------------------------------------------------------------------------------------%

%Nome da Cidade de defesa
\newcommand{\nomeDaCidade}{São Carlos}

%Nome da Universidade Nome da Faculdade ou Instituto - Campus nome da Cidade

\newcommand{\nomeDaUniversidade}{
	Universidade Federal de São Carlos - UFSCAR\\
	\NomeDaFaculdade -- Campus de\ \nomeDaCidade
}


