
%
% A P Ê N D I C E S
%
% ELEMENTO OPCIONAL. Consiste em um texto ou documento elaborado pelo autor a fim de complementar
% sua argumentação. Se necessário ter mais de um apêndice, basta adicionar cada um dentro de um dentro de um comando "\chapter". % Se não for utilizar um Apêndice, basta apagar
% todo o código dessa seção.
%------------------------------------------------------------------------------------%

\begin{apendicesenv}
	
	\chapter{RESUMO DAS REGRAS DE ELABORAÇÃO DO TRABALHO ACADÊMICO }
	\begin{table}[htb!]
		\centering
		\begin{tabular}{l|
				>{\columncolor[HTML]{D3DFEE}}l }
			\hline
			\textbf{Configuração da página} & \begin{tabular}[c]{@{}l@{}}formato A4 (21 cm x 29,7 cm)\\ margens: \\    esquerda e superior: 3 cm\\    direita e inferior: 2 cm\end{tabular}                                                                                                                                                          \\ \hline
			\textbf{Fonte}                  & \cellcolor[HTML]{A7C0DE}\begin{tabular}[c]{@{}l@{}}recomenda-se Arial ou Times New Roman ou similares \\ no caso de software livre, em cor preta\end{tabular}                                                                                                                                          \\ \hline
			\textbf{Tamanho da fonte}       & \begin{tabular}[c]{@{}l@{}}texto geral: 12\\ citações, notas de rodapé, paginação, legendas e fontes \\ das ilustrações e das tabelas: 10\end{tabular}                                                                                                                                                 \\ \hline
			\textbf{Espaçamento}            & \cellcolor[HTML]{A7C0DE}\begin{tabular}[c]{@{}l@{}}texto geral: espaçamento 1,5\\ citações com mais de três linhas, notas de rodapé, ficha \\ catalográfica, natureza, resumo e palavras-chave, legendas \\ e fontes das ilustrações e das tabelas, e referências: \\ espaçamento simples\end{tabular} \\ \hline
			\textbf{Paginação}              & \begin{tabular}[c]{@{}l@{}}contagem: as páginas antes da introdução devem ser \\ contadas sequencialmente, exceto capa e ficha \\ catalográfica, mas não numeradas. \\ numeração: canto superior direito a partir da Introdução, \\ fonte tamanho 10\end{tabular}                                      \\ \hline
			\textbf{Cada artigo}            & \cellcolor[HTML]{A7C0DE}\textbf{Verifique a norma da revista ou NBR 6022}                                                                                                                                                                                                                              \\ \hline
		\end{tabular}
	\end{table}
	
\end{apendicesenv}
