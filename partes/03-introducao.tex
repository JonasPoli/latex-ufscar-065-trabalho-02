 \chapter{Introdução} % ELEMENTO OBRIGATÓRIO

\begin{flushleft}
	A Assistência Técnica e Extensão Rural (ATER) constitui-se como um instrumento essencial para a promoção do desenvolvimento sustentável em territórios rurais, especialmente junto à agricultura familiar. No entanto, a efetividade dessas ações ainda encontra entraves significativos no que diz respeito à acessibilidade, à linguagem técnica utilizada e à adequação cultural dos canais de comunicação empregados. Tais desafios tornam-se ainda mais evidentes quando se considera a pluralidade linguística e os diversos repertórios socioculturais que permeiam os territórios do Brasil rural.
	
	Diante disso, o presente artigo propõe e apresenta o desenvolvimento do sistema \textbf{AgroSíntese}, uma plataforma digital baseada em inteligência artificial (IA), com atendimento via WhatsApp, voltada à oferta de ATER de forma dialógica, personalizada e sensível às particularidades regionais. Essa proposta considera que o Brasil é um território linguisticamente diverso, com zonas dialetais bem definidas conforme mapeado pelo Atlas Linguístico do Brasil (ALiB). Cada uma dessas regiões apresenta variações fonético-fonológicas que, se respeitadas, aumentam significativamente a empatia, a compreensão e a adesão à comunicação técnico-educativa. Assim, os sotaques não são apenas características vocais, mas expressões legítimas de identidades culturais que o AgroSíntese busca preservar e ativar na comunicação com os sujeitos do campo.

	
	A base conceitual do trabalho ancora-se em autores que problematizam a linguagem e o conhecimento como construções sociais e situadas. Bakhtin \cite{bakhtin1997estetica} compreende a linguagem como essencialmente dialógica, defendendo que o sentido de um enunciado só pode ser compreendido dentro de seu contexto social e histórico. A esse entendimento soma-se Freire \cite{freire2013extensao}, ao afirmar que toda comunicação educativa precisa ser horizontal e mediada por uma escuta ativa, onde o educador se reconhece também como aprendiz. Nesse sentido, o AgroSíntese não é apenas uma ferramenta tecnológica, mas um dispositivo de mediação cultural e educativa.
	
	A comunicação rural, compreendida a partir dos estudos de Zuin \cite{zuin2021comunicacao}, exige mais do que a simples presença de canais digitais: demanda uma escuta qualificada, contextualizada e orientada pela compreensão dos processos comunicativos do território. A tecnologia, portanto, deve estar a serviço das relações, e não o contrário. Larrosa \cite{larrosa2014}, ao refletir sobre a experiência na educação, contribui ao destacar que o saber não se dá somente pela informação transmitida, mas pela significação que cada sujeito constrói a partir de seu contexto de vida, sua história e suas práticas.
	
	Neste artigo, descreve-se a concepção, estruturação e objetivos do AgroSíntese, enfatizando sua metodologia de desenvolvimento orientada pela perspectiva da ATER Digital Participativa \cite{parra2022ater}, e refletindo sobre suas contribuições para uma extensão rural verdadeiramente inclusiva, crítica e transformadora. O foco recai sobre o modo como as tecnologias podem contribuir para promover a autonomia dos agricultores familiares, valorizando suas vozes e saberes em um processo comunicativo bidirecional e humanizado.
	
	
	
	
\end{flushleft}


%Em caso de dúvidas, consulte o Manual de normalização de trabalhos acadêmicos: apresentação: ABNT (2023): \url{https://docs.google.com/document/d/1ipkGiAUnAr_YBTrpFJpnud3aa4llBsROpBdkUfw91L0/edit?usp=sharing}
