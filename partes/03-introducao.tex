 \chapter{Introdução} % ELEMENTO OBRIGATÓRIO

\begin{flushleft}
A diversidade linguística das diversas regiões do Brasil é uma das maiores expressões da complexidade cultural do país. Não é um exagero afirmar que essa diversidade espalhada pelo território brasileiro guarda uma memória histórica que não se perdeu, sendo preservada na oralidade do cotidiano.

Esta diversidade linguística brasileira se apresenta não só nos sotaques diferentes de uma região para outra como também de idiomas completos, como falado por algumas tribos indígenas mais isoladas.

Devido a esta complexidade, diversos trabalhos já foram feitos para tentar identificar os idiomas e sotaques falados no Brasil, sendo alguns bem antigos, como os de Antenor Nascentes iniciado em 1922 \cite{nascentes1953}, que organizava a fala brasileira em zonas dialetais, até os trabalhos mais recentes como os do Atlas Linguístico do Brasil (ALiB), que analisam e organizam as diversas formas de se falar tanto o português quanto as outras no Brasil. Agora, com o advento das mais novas mudanças tecnológicas, nos empurra para novos desafios, nem sempre simples de resolver.

Um problema apresentado pelos diversos estudos existentes é a diminuição das diferenças. Cada vez mais, incentivadas ou até mesmo forçadas, as formas de se falar do brasileiro cada vez se tornam mais parecidas uma com as outras.

Para entendermos melhor essa questão, vamos levar em consideração algumas informações levantadas por Gilvan Müller de Oliveira, que identificou que no Brasil, são falados cerca de 210 idiomas. As populações indígenas falam em torno de 170 línguas (conhecidas como autóctones), enquanto comunidades de descendentes de imigrantes mantêm vivas aproximadamente 30 línguas (as chamadas alóctones). Além disso, temos duas línguas de sinais utilizadas pelas comunidades surdas: a Libras (Língua Brasileira de Sinais) e a língua de sinais Urubu-Kaapór. Ou seja, o Brasil é, sim, um país de muitas línguas — plurilíngue — assim como a maioria dos outros países do mundo (94\% dos casos) no Brasil se fala mais de um idioma \cite{noauthor_plurilinguismo_nodate}.

Contudo, os números atuais indicam que a diversidade linguística brasileira vem diminuindo. Conforme apontado por Rodrigues \cite{rodrigues_linguas_nodate}, há pouco mais de cinco séculos, Cabral chegou por aqui, existiam aproximadamente 1.078 línguas distintas em uso e uma população de 5 milhões. Hoje, com cerca de 211 milhões de habitantes, a quantidade de idiomas falados diminuiu para 15\% (210) dessa diversidade linguística original. Estes números indicam que é preciso criar formas de se preservar nossos diversos idiomas.

Isso levanta um problema que é político ao mesmo tempo cultural: como desenvolver tecnologias de comunicação que não reforcem a centralização linguística ainda mais, que não apaguem o regionalismo e, ao invés disso, faça uso dele?

Diante disso, este artigo tem como objetivo propor o uso do sistema de Discagem Direta à Distância (DDD) como uma base territorial para mapeamento de dialetos brasileiros, visando à criação de tecnologias assistivas linguisticamente sensíveis. Não porque o DDD possua alguma ligação linguística direta e sim porque ele já está institucionalizado, é familiar à população e, em certo sentido, traça um mapa funcional do país. Se cruzarmos essa estrutura de códigos com a tabela de dialetos, teremos uma forma de criar tecnologias que se baseiam no DDD para identificar o dialeto predominante da região.

O artigo está organizado da seguinte forma: primeiro, apresentamos o referencial teórico que fundamenta a proposta; em seguida, detalhamos a metodologia utilizada para associar códigos DDD a zonas dialetais; depois, discutimos os resultados obtidos e suas implicações para o desenvolvimento de tecnologias inclusivas; por fim, concluímos com as contribuições e próximos passos desta abordagem.
	
\end{flushleft}


%Em caso de dúvidas, consulte o Manual de normalização de trabalhos acadêmicos: apresentação: ABNT (2023): \url{https://docs.google.com/document/d/1ipkGiAUnAr_YBTrpFJpnud3aa4llBsROpBdkUfw91L0/edit?usp=sharing}
