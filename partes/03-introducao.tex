 \chapter{Introdução} % ELEMENTO OBRIGATÓRIO

\begin{flushleft}
A diversidade linguística brasileira representa um patrimônio cultural inestimável, refletindo a rica história de formação do país e as múltiplas identidades que compõem o mosaico social nacional. Desde os primeiros estudos dialetológicos sistemáticos, como a divisão proposta por Antenor Nascentes em 1922, até as pesquisas contemporâneas do Atlas Linguístico do Brasil (ALiB), os linguistas brasileiros têm se dedicado a mapear e compreender as variações do português falado nas diferentes regiões do país.
Paralelamente, o avanço tecnológico tem transformado profundamente as formas de comunicação e interação social, com impactos significativos em todos os setores da sociedade, incluindo as áreas rurais e comunidades tradicionalmente marginalizadas. Nesse contexto, emerge um desafio crucial: como desenvolver tecnologias de comunicação que respeitem e valorizem a diversidade linguística brasileira, promovendo inclusão digital sem impor um padrão linguístico hegemônico?
Este artigo propõe uma abordagem inovadora para enfrentar esse desafio: utilizar o mapa de Discagem Direta à Distância (DDD) como referência para o mapeamento dos dialetos brasileiros, visando o desenvolvimento de tecnologias assistivas que se adaptem às variações linguísticas regionais. A escolha do sistema de DDD como base para essa proposta justifica-se por sua ampla disseminação e reconhecimento pela população brasileira, além de sua infraestrutura já estabelecida no setor de telecomunicações.
A fundamentação teórica desta proposta baseia-se em duas vertentes complementares: o dialogismo de Mikhail Bakhtin e a pedagogia libertadora de Paulo Freire. De Bakhtin, incorporamos a compreensão da linguagem como fenômeno essencialmente dialógico e social, em que os enunciados são sempre orientados para o outro e carregados de valores culturais e ideológicos. De Freire, adotamos a perspectiva da comunicação como
prática libertadora, que deve reconhecer e valorizar os saberes e modos de expressão locais, promovendo uma relação horizontal e dialógica entre os interlocutores.
No contexto específico da comunicação rural e da Assistência Técnica e Extensão Rural (ATER) digital, essa abordagem ganha relevância adicional, considerando os desafios de estabelecer pontes comunicativas efetivas entre diferentes realidades socioculturais e linguísticas. Como destacam Zuin (2021) e Parra et al. (2021), a comunicação rural eficaz depende fundamentalmente do reconhecimento e respeito às particularidades culturais e linguísticas das comunidades rurais.
Ao propor o uso do mapa de DDD como referência para o mapeamento dos dialetos brasileiros, este trabalho busca contribuir para o desenvolvimento de tecnologias linguisticamente inclusivas, que possam adaptar-se ao modo de falar local, facilitando a comunicação e promovendo o respeito à diversidade cultural. Essa proposta alinha-se à perspectiva da teoria ator-rede de Bruno Latour, ao reconhecer tanto os dialetos quanto os códigos de DDD como atores em uma rede sociotécnica complexa, que influenciam e são influenciados pelas práticas comunicativas e tecnológicas.
Nas seções seguintes, apresentaremos a evolução histórica dos estudos dialetais no Brasil, a estrutura do plano de numeração brasileiro, a metodologia de associação entre DDDs e dialetos, os resultados dessa associação e as implicações teóricas e práticas dessa proposta para o desenvolvimento de tecnologias assistivas linguisticamente inclusivas.

	
	
	
	
\end{flushleft}


%Em caso de dúvidas, consulte o Manual de normalização de trabalhos acadêmicos: apresentação: ABNT (2023): \url{https://docs.google.com/document/d/1ipkGiAUnAr_YBTrpFJpnud3aa4llBsROpBdkUfw91L0/edit?usp=sharing}
