 \chapter{Introdução} % ELEMENTO OBRIGATÓRIO

\begin{flushleft}


A notável diversidade linguística que caracteriza as múltiplas regiões do Brasil configura-se como uma das mais ricas expressões da complexidade cultural da nação. Pode-se afirmar, sem exagero, que essa pluralidade, disseminada por todo o território, encerra uma memória histórica que resistiu ao tempo, preservada fundamentalmente na oralidade do cotidiano. Esta riqueza manifesta-se não apenas através dos distintos sotaques regionais, mas também pela presença de idiomas completos, como os falados por comunidades indígenas mais isoladas e por descendentes de imigrantes.

Diante dessa complexidade, inúmeros trabalhos buscaram identificar e sistematizar os idiomas e variantes falados no Brasil. Destacam-se desde estudos pioneiros, como os de Antenor \cite{nascentes1953}, iniciados em 1922, que propuseram uma organização da fala brasileira em zonas dialetais, até investigações mais recentes e abrangentes, como as direcionadas pelo projeto Atlas Linguístico do Brasil (ALiB). Este último dedica-se à análise e organização das variadas formas de expressão do português e de outras línguas presentes no país \cite{cardoso2014alib, Aguilera2022}.

Contudo, um desafio persistente apontado por diversos estudos é a progressiva diminuição das diferenças linguísticas. As formas de falar do brasileiro, por vezes incentivadas ou mesmo forçadas por dinâmicas sociais e culturais, tendem a uma crescente homogeneização. Para dimensionar a questão dos sotaques, é importante analisar um fenômeno parecido que afeta os idiomas no Brasil. Gilvan Müller de Oliveira identificou que, atualmente, são falados cerca de 210 idiomas no país. As populações indígenas utilizam aproximadamente 170 dessas línguas (conhecidas como autóctones), enquanto comunidades de descendentes de imigrantes mantêm vivas em torno de 30 línguas (denominadas alóctones). Adicionalmente, existem duas línguas de sinais empregadas pelas comunidades surdas: a Língua Brasileira de Sinais (Libras) e a língua de sinais Urubu-Kaapór. Estes números confirmam o Brasil como um país multilíngue, como acontece com a maioria das nações globais \cite{noauthor_plurilinguismo_nodate}.

Apesar dessa riqueza, os dados históricos indicam uma perda significativa dessa diversidade. Conforme apontado por \cite{rodrigues_linguas_nodate}, há pouco mais de cinco séculos, existiam aproximadamente 1.078 línguas distintas em uso. Hoje, mesmo com um expressivo aumento populacional, o número de idiomas falados reduziu-se drasticamente. Essa constatação deixa claro a urgência na criação de ferramentas para a preservação da diversidade linguística que ainda temo. Além do valor cultural da própria preservação da forma de falar; ainda torna necessário diante dos riscos de exclusão digital por parte de alguns e da perpetuação de vieses linguísticos em tecnologias emergentes, como a inteligência artificial. A ausência de tecnologias linguisticamente conscientes pode agravar desigualdades, dificultando o acesso à informação e a serviços por parte de falantes de uma variedades linguística "não padrão".

Este cenário levanta uma questão fundamental de natureza política e cultural: como desenvolver tecnologias de comunicação que não apenas evitem o reforço da centralização linguística e o apagamento dos regionalismos, mas que, ao contrário, valorizem e utilizem essa diversidade como uma ferramenta?

Diante disso, este artigo tem como objetivo central propor e analisar a viabilidade da utilização do sistema de Discagem Direta à Distância (DDD) como uma base territorial para o mapeamento de dialetos brasileiros. A finalidade é subsidiar a criação de tecnologias assistivas linguisticamente sensíveis e adaptativas. A escolha do DDD não se deve a uma suposta ligação linguística, mas à sua condição de sistema já institucionalizado, familiar à população e que, de certa forma, delineia um mapa funcional do país. Ao cruzar essa tabela de dialetos de o mapa do plano de numeração brasileiro, busca-se um método prático para que tecnologias possam, a partir do DDD, supor o perfil dialetal predominante de uma região.

O artigo está organizado da seguinte forma: inicialmente, apresenta-se o referencial teórico que fundamenta a proposta. Em seguida, detalha-se a metodologia empregada para associar os códigos DDD às zonas dialetais. Posteriormente, discutem-se os resultados obtidos e suas implicações para o desenvolvimento de tecnologias inclusivas. Por fim, conclui-se com as contribuições e os próximos passos desta abordagem.



\end{flushleft}


%Em caso de dúvidas, consulte o Manual de normalização de trabalhos acadêmicos: apresentação: ABNT (2023): \url{https://docs.google.com/document/d/1ipkGiAUnAr_YBTrpFJpnud3aa4llBsROpBdkUfw91L0/edit?usp=sharing}
