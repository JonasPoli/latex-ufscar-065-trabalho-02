 \chapter{Introdução} % ELEMENTO OBRIGATÓRIO

\begin{flushleft}
A diversidade linguística no Brasil é uma das expressões mais vívidas da complexidade cultural do país. Não é exagero dizer que essa pluralidade de falares espalhados pelo território nacional guarda, em si, uma memória viva das múltiplas histórias do país. De certa forma, cada sotaque, cada vocabulário regional, cada entonação diferente carrega algo de ancestral, algo que escapou das grandes narrativas e se preservou na oralidade do dia a dia.

Desde os trabalhos mais antigos, como os de Antenor Nascentes em 1922, que iniciou a organização da fala brasileira em zonas dialetais, até os esforços mais recentes do ALiB, o Atlas Linguístico do Brasil, há uma longa caminhada de escuta, categorização e, às vezes, até de perplexidade diante da riqueza de como o português se apresenta por aqui. Se, por um lado, temos essa documentação detalhada da variação linguística, por outro, a velocidade das mudanças tecnológicas nos empurra para desafios novos, nem sempre simples.

Os meios de comunicação e as formas de se comunicar evoluíram. O modo como as pessoas interagem, especialmente com a chegada da internet em áreas mais remotas, vem alterando também as expectativas sobre como se deve falar, escrever ou se fazer entender. Isso acaba impactando especialmente àquelas comunidades mais distantes, que ainda falam de jeitos que, em muitos casos, não se encaixam nas normas mais padronizadas.

Aí surge uma questão que é profundamente política e cultural: como criar tecnologias de comunicação que não reforcem ainda mais a centralização linguística, que não apaguem essas falas regionais, mas, ao contrário, que se moldem a elas? Como construir sistemas que abracem, e não filtrem, essa variedade?

Uma das ideias que propomos aqui é usar o próprio sistema de Discagem Direta à Distância, o DDD, como uma espécie de espelho territorial da linguagem. Não porque o DDD tenha alguma relação linguística direta, mas porque ele já está estabelecido, é conhecido pela população e, de certa forma, desenha um mapa funcional do país. Se cruzarmos esses códigos com os dados dialetológicos, talvez possamos encontrar uma forma prática de associar regiões a determinadas formas de fala. E aí, a partir disso, desenvolver tecnologias que saibam escutar e responder em sintonia com esses modos de dizer.

Claro, essa proposta não parte do zero. Ela se apoia em dois pensadores que ajudam a costurar a base teórica dessa reflexão: Mikhail Bakhtin e Paulo Freire. De Bakhtin, nos interessa a ideia de que toda linguagem é, no fundo, um diálogo — ninguém fala para o vazio, e todo enunciado carrega marcas de outras vozes, de outros discursos, até mesmo de conflitos. Já Freire nos convida a olhar a comunicação como um ato de libertação — e não como uma transmissão unilateral. É preciso escutar o outro, respeitar seus saberes, sua forma de dizer, sua experiência concreta no mundo.

E talvez esse ponto seja ainda mais importante quando falamos da comunicação rural — especialmente quando se trata da ATER digital, a Assistência Técnica e Extensão Rural feita por meio de plataformas, apps, vídeos, mensagens... Como se conecta, de fato, com pessoas que vivem e falam de forma tão particular? Zuin e Parra, por exemplo, já apontaram em 2021 que qualquer ação comunicativa no campo só funciona se partir do reconhecimento dessas diferenças culturais e linguísticas.

Assim, quando propomos usar o DDD como uma chave para pensar os dialetos e moldar a tecnologia a partir disso, não estamos apenas fazendo um exercício técnico — estamos propondo um gesto de escuta. Um gesto que, de certa forma, também dialoga com a teoria ator-rede de Bruno Latour, onde tanto os códigos de DDD quanto os dialetos não são só ferramentas ou dados, mas elementos ativos numa rede em que tudo influencia tudo: o modo de falar influencia o design da ferramenta, que por sua vez muda a forma como se fala, e por aí vai.

Nas próximas partes deste trabalho, vamos entrar um pouco mais nos detalhes. Primeiro, voltamos à história dos estudos dialetológicos no Brasil, para entender de onde vêm essas classificações. Depois, passamos pelo funcionamento do sistema de DDD — que, por mais banal que pareça, tem sua lógica própria. Em seguida, explicamos como estamos tentando cruzar essas duas dimensões: códigos de área e variações linguísticas. E por fim, discutimos as possíveis implicações disso — tanto no campo da teoria quanto, principalmente, no desenvolvimento de tecnologias que não só funcionem, mas que funcionem bem justamente por respeitarem as maneiras diferentes de falar deste país tão... falante.


	
\end{flushleft}


%Em caso de dúvidas, consulte o Manual de normalização de trabalhos acadêmicos: apresentação: ABNT (2023): \url{https://docs.google.com/document/d/1ipkGiAUnAr_YBTrpFJpnud3aa4llBsROpBdkUfw91L0/edit?usp=sharing}
