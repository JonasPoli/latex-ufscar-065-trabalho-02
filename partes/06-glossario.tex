
%--------------------------------------------------------------------------------- 
%GLOSSÁRIO 
%
% ELEMENTO OPCIONAL. É uma lista de termos, palavras ou expressões técnicas utilizadas no texto e acompanhadas de seus respectivos significados, ordenada alfabeticamente. deve iniciar em página distinta, logo após a bibliografia consultada, se houver; deve conter o título GLOSSÁRIO, centralizado, em letras maiúsculas, e sem indicativo numérico.

%usar esses comandos para incluir o glossário e que apareça no sumário: \clearpage   %\chapter*{GLOSSÁRIO} \addcontentsline{toc}{chapter}{GLOSSÁRIO} 
%Se não for utilizar um Glossário , basta apagar todo o código dessa seção.
%-----------------------

\clearpage
\chapter*{GLOSSÁRIO} \addcontentsline{toc}{chapter}{GLOSSÁRIO} 

\noindent\textbf{Acervo:} conjunto de bens que integram o patrimônio de um indivíduo, de uma instituição, de uma nação, agrupados por atribuição de valor, segundo sua natureza cultural e seguindo uma lógica de organização.\\ \\
\textbf{Acessibilidade:} facilidade no acesso ao conteúdo e ao significado de um objeto digital.\\ \\
\textbf{Acesso aberto:}  refere-se à disponibilidade e acesso gratuito por qualquer pessoa aos resultados de pesquisas científicas. Baseia-se na premissa de que o conhecimento científico é um bem público e, portanto, deve estar disponível a todos.\\ \\
\textbf {Direito autoral ou direito de autor}: é um conjunto de privilégios conferidos por lei à pessoa física ou jurídica criadora da obra intelectual, para que ela possa usufruir de quaisquer benefícios morais e patrimoniais resultantes da exploração de suas criações.\\ \\
\textbf {Formato de arquivo:} atributo de um arquivo que descreve sua codificação e identificado pela extensão no final do nome do arquivo. Por exemplo: *.DOC, *.PDF, *.JPEG.\\ \\
\textbf {Metadados:} Dados estruturados que descrevem e permitem encontrar, gerenciar, compreender e/ou preservar documentos ao longo do tempo.\\ \\
\textbf {Repositório Institucional:} sistema de informação que visa armazenar, preservar, organizar, disseminar e promover acesso aberto à produção intelectual produzida nas instituições de ensino e pesquisa.\\ \\
\textbf {Submissão:} é o ato de entregar um documento técnico-científico no RI e/ou BD.

