
\section{ILUSTRAÇÕES}

São consideradas ilustrações: desenhos, esquemas, fluxogramas, fotografias, gráficos, mapas, organogramas, plantas, quadros, retratos, figuras, imagens, entre outros. Veja exemplo \ref{figura:ex}:
% Para que a ilustração fique próxima ao texto a que se refere, para isso, dentro do ambiente figure, inclua o comando \label{nome da figura} e no texto indique \ref e selecione o nome da figura. 
% ORIENTAÇÃO: já foi pré definido em \latex  Parte superior: 
% Formatação: fonte tamanho 12, espaçamento 1,5
% Parte inferior: 
% Formatação: fonte tamanho 10, espaçamento simples.  Atenção, não esquecer de incluir a fonte consultada, mesmo que tenha sido elaborada pelo autor do trabalho            
\begin{figure}[h]
	\centering
	\caption{Banner da Semana do Livro e das Bibliotecas da Unesp 2022: Centenário da Semana de Arte Moderna }
	\includegraphics[width=1.0\linewidth,scale=1.0]{images/banner-semana.png}
	\par
	\raggedright
	{\small Fonte: \citeonline{banner}}
	\par
	{\small Legenda: Fundo bege com logo em cinza e vermelho. Com escritos em cinza: Semana do Livro e das Bibliotecas da Unesp, escrito em vermelho: 2022 Centenário da semana de arte moderna. }
	\label{figura:ex}
\end{figure}

ATENÇÃO: Existem muitas formas diferentes de formatar um elemento utilizando os recursos do \LaTeX, o importante é manter a descrição, a referência e a formatação do elemento gráfico conforme o padrão. Observe o exemplo no \ref{quadro:ex}.

